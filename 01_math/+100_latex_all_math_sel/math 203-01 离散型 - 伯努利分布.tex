\documentclass[UTF8]{ctexart}

\usepackage{subfiles}  

%下面的语句, 引入你的头部设置文件
\usepackage{C:/phpStorm_proj/02_myself_ID_EGO/+100_latex_all_math_sel/myPreamble} 
%必须是绝对路径,才能让各个tex在单独编译时使用到

\title{文件名}


%---------------------------------


\begin{document}
	\tableofcontents % 生成目录
	\date{} % 若不写这句, 则默认也会渲染出日期, 所以我们要手动赋空值
	\maketitle  %这行代码, 让你前面的 title, author, date生效
	
	
	
	\section{伯努利模型 bernoulli model}
	
	- 独立试验序列 :  \\
	在相同的试验条件下,进行一系列随机试验 $ E_1, E_2, ... E_n$, (每次做的实验, 可以是不相同的), 观察某事件A发生与否.若每次试验结果相互独立, 则这样的一系列试验称为``独立试验序列". \\
	
	- n重独立试验 : \\
	把一个试验, 重复做n次. 即:   E, E, ... E, 记作: $ E^n$ \\
	
	
	
	\subsection{伯努利试验 : 其试验结果只有两种: 成功, 失败}
	
	伯努利试验 :  \textbf{其试验结果只有两种.} 即: $ \Omega = \{A, \overline {A}\}$ \\
	
	属于``伯努利试验"的例子有: \\
	- 掷硬币, 结果只有``正面"和``反面"两种. \\
	- 射击, 结果只有``击中"和``没击中"两种.  \\
	- 检验产品, 结果只有``合格",``次品" 两种. \\
	
	不属于``伯努利试验"的例子是 : 掷骰子, 有6种结果. \\
	
	如果在一个试验中, 我们只关心某个事件A 发生与否, 那么就称这个试验为``伯努利试验". 此时, 试验的结果可以看成只有两种: A发生, 或 A不发生.  相应的数学模型, 就称为``伯努利概型". \\
	
	
	\subsection{ n重伯努利试验 : 事件A恰好发生k次的概率,就是= }
	
	n重伯努利试验 : 	就是把``伯努利试验"重复做n次, 每次都是独立的. 并且试验结果只有两种. \\
	比如, 抛硬币, 是一个伯努利试验 (它只有正面, 反面, 这两种结果). 我们做100次这个试验, 就是做了 100重伯努利试验. \\
	
	设在单次试验中, 事件A发生的概率为P, 将此试验重复独立地进行n次, 则事件A恰好发生k次的概率是多少? 通常记这个概率为: $ P_n(k), \quad  k= 0,1,2,...,n$.
	
	
	
		
	
	\section{离散型 : 伯努利分布}
	
	
	
	
	
	
\end{document}