\documentclass[UTF8]{ctexart}

\usepackage{subfiles}  

%下面的语句, 引入你的头部设置文件
\usepackage{C:/phpStorm_proj/02_myself_ID_EGO/+100_latex_all_math_sel/myPreamble} 
%必须是绝对路径,才能让各个tex在单独编译时使用到

\title{文件名}


%---------------------------------


\begin{document}
	\tableofcontents % 生成目录
	\date{} % 若不写这句, 则默认也会渲染出日期, 所以我们要手动赋空值
	\maketitle  %这行代码, 让你前面的 title, author, date生效
	
	
	
	
	
	
	\part{随机变量函数的分布}
	
	意思就是说, 假如我们已经知道 某个X 是某种类型的分布了, 比如 X 它是几何分布的, 二项分布的等. 则进一步, 而我们还想知道, 用这个X 来构造出的其他函数, 会是什么类型的分布呢? 比如, Y=3X-5,  这个Y是由X构造出来的, 那么这个Y, 也是和X相同类型的分布吗? 还是说, Y是其他类型的分布? \\
	
	→ 随机变量X, 它取x时, 其``累加函数"是: $\boxed{F_X (x)=P\{X \leq x\}}$ \\	
	
	→ 由随机变量X, 构造出的一个新 Y (比如 Y=``多少倍的X, 再加上某个数"之类), 这个Y 的``累加函数", 是: $\boxed{F_Y(x)=P\{Y \leq x\}}$  	← 等号左边的 $F_Y(x)$ 意思是: Y是从X构造出来的.  累加函数(用F表示), 所以 $F_Y(x)$ 就是指 ``由X构造出来的新的Y"的累加函数. \\
	
	
	
	
	
\end{document}