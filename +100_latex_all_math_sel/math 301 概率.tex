\documentclass[UTF8]{ctexart}

\usepackage{subfiles}  

%下面的语句, 引入你的头部设置文件
\usepackage{C:/phpStorm_proj/02_myself_ID_EGO/+100_latex_all_math_sel/myPreamble} 
%必须是绝对路径,才能让各个tex在单独编译时使用到

\title{概率}


%---------------------------------


\begin{document}
	\tableofcontents % 生成目录
	\date{} % 若不写这句, 则默认也会渲染出日期, 所以我们要手动赋空值
	\maketitle  %这行代码, 让你前面的 title, author, date生效
	
	
	- 事件 : 即每种结果, 就叫一个``事件".
	
	- 基本事件 : 不能再分解的结果, 就称为``基本事件".
	
	- 复合事件 : 由``基本事件"复合而成. \\
	
	- 样本全集, 或样本空间, 即所有``基本事件"的集合. 用 Ω 表示.  
	
	如: 扔两个硬币, 其结果``样本空间"就是: Ω={(正,正), (正, 反), (反,正), (反,反)}
	
	- 样本空集, 用 Φ 表示 \\
	
	- 样本点 : 就是``样本空间"中的元素, 即"基本事件". 用 ω 表示
	
	- 无限可列个 : 意思就是: 能按某种规律, 排成一个序列. 如:
	
	
	
	
	
	
	
	
\end{document}