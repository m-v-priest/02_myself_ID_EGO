\documentclass[UTF8]{ctexart}

\usepackage{subfiles}  

%下面的语句, 引入你的头部设置文件
\usepackage{C:/phpStorm_proj/02_myself_ID_EGO/+100_latex_all_math_sel/myPreamble} 
%必须是绝对路径,才能让各个tex在单独编译时使用到

\title{文件名}


%---------------------------------


\begin{document}
	\tableofcontents % 生成目录
	\date{} % 若不写这句, 则默认也会渲染出日期, 所以我们要手动赋空值
	\maketitle  %这行代码, 让你前面的 title, author, date生效
	
	
	
	\section{ 随机变量 random variable}
	
	随机变量: 常用大写字母X,Y,Z 或希腊字母来表示. \\
	随机变量的取值 : 用小写字母 x,y,z等表示. \\
	
	比如, 随机变量X,  其=a的话, 我们就把这个事件记作 $\{X=a\}$.  其概率就是 $ P\{X=a\}$, 或写作 $P(X=a)$. \\
	
	
	\begin{tabular}{|p{0.2\textwidth}|p{0.8\textwidth}|}
		\hline
		离散型随机变量 &  其值是 ``有限可列举"的, 如, 为n个有限取值:$ \{x_1, x_2, ... , x_n\}$).\\
		\hline
		连续型随机变量 &  \makecell[l]{ 其值无法逐个列举(即是无穷无尽个的), 是一段区间. \\ 只能写成: $X=\{x | a \leq x \leq b\}, -\infty < a < b < \infty $. } \\
		\hline
	\end{tabular}
	
	
	
	
	
	
	
	
	
\end{document}