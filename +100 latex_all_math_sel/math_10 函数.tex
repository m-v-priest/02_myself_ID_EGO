\documentclass[UTF8]{ctexart}

\usepackage{picinpar, graphicx} % 导入这个库后,就能支持插入表格
\graphicspath {{img_math/},{img2/}} %图片目录在当前目录的 img 和 img2文件夹下

\usepackage{algorithm, algorithmic,amsmath, amssymb,bm} % 支持数学公式输入
\usepackage{ctex} % 支持字体加粗效果, 代码为 \textbf{加粗}

\usepackage{multicol} %用于实现在同一页中实现不同的分栏
\usepackage{wrapfig} %用于实现图文混排
\setlength{\parindent}{0pt} % 放在段首,之后的所有段落都将取消首行缩进

% 页面边距设置
\usepackage{geometry} %导入版面设置的宏包
\geometry{left=1.5cm, right=1.5cm, top=2cm, bottom=2cm} % 使用命令:\geometry{left=左边距,right=右边距,top=上边距,bottom=下边距}

\usepackage[skins]{tcolorbox} % 导入该包, 才能支持彩色文本框效果.  必须标注skin,才能使用shadow命令显示阴影


\title{函数}


\begin{document}
	\tableofcontents % 生成目录
	\maketitle  %这行代码, 让你前面的 title, author, date生效

\part{反函数}



\begin{tabular}{|l| l| }
	\hline
	函数f 是: 输入x, 输出y.
	 &  f(x自变量) = y因变量. \\	 
	\hline	
	
	反函数 $f^{-1}$ 是: 输入y, 输出x. 
	&  相当于时间倒流, 把原函数的功能倒过来. 就像线性代数中的"逆矩阵"变换功能.\\
	\hline
\end{tabular}
\\

``反函数"和``原函数", 图象关于直线 y=x 对称. \\


\begin{tcolorbox}[title = {例},boxrule={0.1em},colframe={black!10}, colback={black!3},colbacktitle={black!10},coltitle={black}]
有函数 y = 3x+5, 即输入x, 输出y. 它可以变为: 
	\begin{align*}
		& 3x = y-5 \\
		& x = \frac{y-5} {3} 
	\end{align*}
	
	这样, 就是输入y, 输出x 的形式了, 即就变成了``反函数".
	
	但一般我们习惯于将输入值, 用x表示; 输出值 , 用y值表示, 所以上面的反函数, 就索性写成  $ y = \dfrac{x-5} {3} $, 但你不要混淆这里的x和y的意义. 这里的x是原y值, 这里的y是原x值.
\end{tcolorbox}


~\\
\hrule
~\\


\part{初等函数}

\section{power function 幂函数:  $ y = x^{exp} $}

变量x 作为``底"的, 就是幂函数. 形如 $ y=x^2 $ , 格式是 $ y = x^{exp} $


~\\
\hrule
~\\


\section{exponential function 指数函数:  $ y=base^x $}

变量x 在肩膀上做为次方来用的, 就是``指数函数". 形如 $ y=100(1+0.1)^x $. 格式是   $ y=base^x $. 其中, base \textgreater 0 并且 base≠1.

其实, ``投资回报率"终值计算公式 $ F=P(1+i)^n $, 就是指数函数. 如: $ y=100(1+0.1)^x $ 


~\\
\hrule
~\\


\section{trigonometric function 三角函数:  $ y=base^x $}

\subsection{sin \& arcSin}





\subsection{cos \& arcCos}



\subsection{tan \& arcTan}




\subsection{cot \& arcCot}




\subsection{sec \& arcSec}




\subsection{csc \& arcCsc}



\end{document}

